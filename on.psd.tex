% FINAL VERSION
\documentclass[final,a4paper]{IEEEtran}
\newcommand{\myfigwidth}{3in}

%########Document Definitions



\usepackage[final]{graphicx}
\usepackage{flafter}
\usepackage{amssymb,amsmath,amsfonts}
\interdisplaylinepenalty=2500
\usepackage{cite}
\usepackage{cases}


%-------------------------------------------------------
% Custom Commands
%-------------------------------------------------------
\newcommand{\code}[1]{{\small \tt #1}}
\newcommand{\eqn}[1]{
\begin{equation}
#1
\end{equation}}
\newcommand{\eqna}[1]{
\setlength{\arraycolsep}{0.0em}
\begin{eqnarray} 
#1
\end{eqnarray}
\setlength{\arraycolsep}{5pt}
}


\newcommand{\dg}{}
\newcommand{\dge}{G_e}
\newcommand{\dgc}{G_c}
\newcommand{\dgcd}{\xi_e }
\newcommand{\rgc}{R[\dgc]}
\newcommand{\req}[1]{(\ref{eq:#1})}
\newcommand{\mean}[1]{\frac{1}{N}\sum^N{#1}}
\newcommand{\tabsp}{\\[6pt]}
\newcommand{\en}[1]{\overline{e_{#1}^2}}
\newtheorem{theorem}{Theorem}
\newcommand{\powavg}[1]{\lim_{T\to\infty} \frac{1}{T}\int^{+T/2}_{-T/2}{ #1 dt}}
%\newcommand{\inti}[1]{\int_{-\infty}^{\infty}{#1}}
\newcommand{\eomp}[1]{e^{j \omega #1}}
\newcommand{\eomn}[1]{e^{-j \omega #1}}
\newcommand{\inti}[1]{\int_{-\infty}^{\infty}{#1}}
\newcommand{\intoi}[1]{\int_{0}^{\infty}{#1}}
\newcommand{\intio}[1]{\int_{-\infty}^{0}{#1}
}
\newcommand{\eqaa}{&{}={}&}

\bibliographystyle{IEEEtran}

\author{Carsten~Wulff}

\title{On Spectral Densities}
\begin{document}
\maketitle

\begin{abstract}
Spectral density is a measure on the average
power of a signal as a function of frequency. There are two different
definitions used in the literature. This paper serves to underline
the difference between them. We also give the origin of spectral density.
%Spectral densities, or spectral intensities, is a useful mathematical
%tool to describe power content at different frequencies. It is used to
%describe the noise power of noise sources like thermal noise, shot
%noise and flicker noise. However, if one look in the literature the
%definitions of spectral density can be confusing. This short story
%aims to clarify some of the sources the definitions of spectral density.
\end{abstract}

Warning: This is not an introduction to spectral density. If the
subject is completely unfamiliar I'd advise reading another source. For
example chapter 4 in \cite{johns} or chapter 7 in \cite{razavi}.

\section{Definition of Spectral Density}

There are two different definitions of spectral density used in the literature. They differ by a factor of two.
The one used in signal processing books, like  \cite{gray.r.m}, is 
\eqn{
\label{eq:psd1}
S_{x1}(f) = \inti{R_{x1}(\tau)e^{-j\omega\tau}d\tau}
}
And the one often used in books about noise, like \cite{ziel}, is
\eqn{
\label{eq:psd2}
S_{x2}(f) = 2\inti{R_{x2}(\tau)e^{-j\omega\tau}d\tau}
}
In both cases $R_{xi}(\tau)$ is the auto-correlation function defined as
\eqn{
R_{xi}(\tau) = \overline{x_i(t)x_i(t+\tau)}
}
As we can plainly see 
\eqn{S_{x1}(f) \neq S_{x2}(f)}
, there is no way these two can be made equal if 
\eqn{\label{eq:rxequal}R_{x1}(\tau) = R_{x2}(\tau)}
This is ok, there is no
problem having two different definitions for two different
functions. In reality $S_{x1}(f)$ and $S_{x2}(f)$ are different functions of
frequency, and we could say that
\eqn{S_{x2}(f) = 2S_{x1}(f)} if \req{rxequal} is true.  

\section{Sources of Confusion}
The problem with spectral density arises when reading literature from
different communities, for example \cite{gray.r.m} and \cite{ziel} where
$S_x(f)$ is used for both $S_{x1}(f)$ and $S_{x2}(f)$. 
When I started investigating spectral densities this lead me to
believe that different sources defined the same measure
``spectral density'' in two different ways. The more sources I
investigated the more unsure I was about which of the two definitions
that was correct. After months of searching (not actively, but
sporadicly) I eventually found the original source of the definition
of spectral density \cite{einstein14}. Having the original source
helped, but I still don't know when the original definition split into \req{psd1} and
\req{psd2}. However, I'm pretty sure the it's just a matter of
convenience. To see why \req{psd2} is the most common among sources
concerning noise we look at the inverse Fourier Transform.
By the way, if you had not
noticed yet, \req{psd1} says that {\em Spectral density is the Fourier
Transform of the Auto-Correlation function}.
The inverse Fourier Transform of \req{psd1} is
\eqn{
R_{x1}(\tau) = \frac{1}{2\pi}\inti{S_{x1}(f)e^{j\omega\tau}dw} = \inti{S_{x1}(f)e^{j\omega\tau}df}
}
,since $dw = df dw/df = 2\pi df$. And for \req{psd2}
\eqn{
R_{x2}(\tau) = \frac{1}{2}\inti{S_{x2}(f)e^{jw\tau}df}
}
Before we proceed lets get rid of the $e$'s. We know that $e^{j\alpha} =
\cos \alpha + j \sin \alpha$. So we could rewrite \req{psd1} as
\eqn{
S_{x1}(f) = \inti{R_{x1}(\tau)[\cos(\omega \tau) + j \sin( \omega
  \tau)]d\tau}
}
and it turns out that since $R_{x1}(\tau)$ is an even function we can
drop the $j\sin{\omega \tau}$ term. $S_{x1}(f)$ is also an even function
since the Fourier Transform of an even function is  even. 

The definitions then become
\eqna{
S_{x1}(f) &{}={}& \inti{R_{x1}(\tau)\cos(\omega\tau)d\tau}\nonumber\\
R_{x1}(\tau) &{}={}& \inti{S_{x1}(f)\cos(\omega\tau)df}
}
and
\eqna{
S_{x2}(f) &{}={}& 2\inti{R_{x2}(\tau)\cos(\omega\tau)d\tau}\nonumber\\
R_{x2}(\tau) &{}={}& \frac{1}{2}\inti{S_{x2}(f)\cos(\omega\tau)df}
}
We can rewrite $R_{x2}(\tau)$ as 
\eqn{
R_{x2}(\tau) = \overline{x_2(t)x_2(t + \tau)} = \int_0^{\infty}{S_{x2}(f)\cos(\omega\tau)df}
}
and if $\tau = 0$
\eqn{
\overline{x_2^2(t)} = \int_0^{\infty}{S_{x2}(f)df} 
}
So using spectral density definition \req{psd2} we see that average
power (mean square value of $x_2(t)$) is equal to the integral from 0 to
infinity of the spectral density. If we use \req{psd1} average power
would be
\eqn{
\overline{x_1^2(t)} = 2\int_0^{\infty}{S_{x1}(f)df} 
}
But if $R_{x1}(\tau) = R_{x2}(\tau)$ then 
\eqn{
\overline{x_2^2(t)} = \overline{x_1^2(t)}
}
even though $S_{x1}(f) \neq S_{x2}(f)$.

Definition \req{psd1} is  called the two-sided spectral density
and \req{psd2} is called the one-sided spectral density.

\section{Example: Thermal Noise}
The spectral density of thermal noise in electronic circuit should be
known to anyone that has studied analog electronics. We normally
define the voltage spectral density of thermal noise as
\eqn{
\label{eq:othermal}
S_{th}(f) = 4kTR
}
where k is Boltzmann's constant, T the temperature in Kelvin and R the
resistance. But \req{othermal} is the spectral density when it is
defined as in \req{psd2}. If we were to use \req{psd1} then the
spectral density of thermal noise would be
\eqn{
S_{th}(f) = 2kTR
}
Both these spectral densities would give the same average power
value if we use the inverse Fourier Transform of \req{psd1} and
\req{psd2}.\footnote{Note that if you calculate the average power of
  $S_{th}(f)$ you'll get infinity. You have to include the bandwidth of
  the circuit you are considering for average power to have a finite value.}

\section{Einstein: The source}
In his 1914 paper \cite{einstein14} Albert Einstein described,
supposedly for the first time, the auto-correlation function and
what we have come to know as the spectral density. He defined the
auto-correlation function as
\eqn{
 \mathfrak{M} (\Delta) = \overline{F(t)F(t + \Delta)}
}
and the intensity (spectral density) as
\eqn{
I(\theta) =  \int_0^{T}{\mathfrak{M}(\Delta) \cos ( \pi \frac{\Delta}{\theta})d\Delta}
}
,where the period $\theta = T/n$ and $T$ is a very large value. The
paper is very short, only 1 page, but it is worth reading. Note that
\req{psd1} is often referred to as the {\em Wiener-Khintchine}  theorem.


\bibliography{IEEEabrv,/Users/wulff/data/2023/aic2023/pdf/phd,/Users/wulff/data/2006/the.converter.collection/Analog-to-Digital/IEEE_J_JSSC}
%\bibliography{IEEEabrv,/home/wulff/svnwork/wulff/work/ntnu/phd/phd}

\end{document}
